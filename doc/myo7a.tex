\documentclass[a4paper,11pt,notitlepage]{article}
\usepackage[utf8]{inputenc}
\usepackage[finnish]{babel}
\usepackage{appendix}
\usepackage{graphicx}
\usepackage{xurl}
\usepackage{hyperref}
\usepackage{booktabs}
\usepackage[labelfont=bf,labelsep=period]{caption}
\usepackage[style=chem-acs,articletitle=true,maxbibnames=6]{biblatex}
\addbibresource{biblio.bib}
\hyphenation{ho-lo-ent-syy-min sek-vens-sien}

\title{\textit{\normalsize MOLE-211 Bioinformatiikka 1: Harjoitustyö}\\ \textbf{MYO7A}}
\author{Kristian Ojala 017226165}
\date{}

\begin{document}
\maketitle

\renewcommand{\abstractname}{Tiivistelmä}
\begin{abstract}
\textit{MYO7A} on geeni, jonka toiminta on edellytys normaalille näölle, kuulolle ja tasapainoaistille. Sen tuote on moottoriproteiini, jota tarvitaan solun sisäisessä kuljetuksessa ja sitä esiintyy erityisesti värekarvojen yhteydessä. Tässä työssä perehdyttiin biotietokantojen avulla \textit{MYO7A}:n proteiinituotteeseen ja geeniontologioihin sekä toteutettiin yhden geenin fylogeneettinen analyysi, johon sisältyi ortologien etsintä, usean sekvenssin rinnastus ja suurimman uskottavuuden puun muodostus. 
\end{abstract}

\section{Johdanto}
Myosiini VIIa on epätyypillisiin myosiineihin kuuluva ATP-riippuvainen aktiinia sitova moottoriproteiini, jota koodaa \textit{MYO7A}-geeni. Sitä ilmennetään verkkokalvon pigmenttiepiteelissä, valoreseptorisoluissa, kuuloaistin karvasoluissa sekä tietyissä kiveksen, keuhkon ja munuaisen soluissa \cite{hasson1995,liu1997}. On osoitettu, että MYO7A:ta tarvitaan muun muassa opsiinin kuljetuksessa valoreseptorin yhdysvärekarvan läpi \cite{liu1999}, näkösykliin osallisen RPE65-entsyymin kuljetuksessa \cite{lopes2011} ja spermatidien kuljetuksessa siemenepiteelissä \cite{wen2019}. Mutaatiot \textit{MYO7A}:ssa voivat aiheuttaa peittyvästi autosomissa periytyvän tyypin 1B Usherin oireyhtymän (USH1B) tai sensorineuraalisen huonokuuloisuuden, joka voi periytyä vallitsevasti (DFNA11) tai peittyvästi (DFNB2) \cite{riazuddin2008}. USH1B ilmenee varhaisessa iässä vaikeana tai erittäin vaikeana kuulovikana, verkkokalvon pigmenttirappeumana ja tasapainoelimen toimintahäiriönä.

\textit{In vitro} biokemialliset kokeet ovat osoittaneet, että MYO7A eli raskasketju ei ole stabiili molekyyli sellaisenaan, vaan se muodostaa stabiilin holoentsyymin yhdessä kevytketjuyksiköiden kanssa \cite{hollo2023}. Kokeissa on selvitetty, että säätelevä kevytketju (RLC), kalmoduliini ja kalmoduliinin kaltainen proteiini 4 (CALML4) puhdistuvat yhdessä raskasketjun kanssa, ja että kalmoduliini dissosioituu kalsiumin vaikutuksesta, mikä on mahdollisesti keino säädellä holoentsyymin mekaanista toimintaa vasteena $\textrm{Ca}^{2+}$-signaloin\-tiin. Lisäksi monomeerinen holoentsyymi omaksuu itseään estävän konformaation, jossa häntä taipuu ja kiinnittyy moottoriosaan heikentäen sen entsymaattista aktiivisuutta. Myosiinin liikkuminen aktiinisäikeitä pitkin edellyttääkin sen dimerisaatiota tai kiinnittymistä sovitinproteiineihin, kuten MyRIP, joita tarvitaan kuorman sitomisessa \cite{hollo2023}. 

MYO7A:n vaihtoehtoisella silmukoinnilla tuotettujen isoformien ATPaasi-aktiivisuudessa ja liukumisnopeudessa aktiinia pitkin on havaittu eroja \cite{hollo2023}. Simpukan ulommissa karvasoluissa ilmennetään kahta isoformia, joita erottaa 11 aminohapon N-terminaalinen jatke. Pidempää ja nopeampaa isoformia (NM\_000260) ilmennetään enemmän simpukkatiehyen kärjessä ja se vähenee tyveä kohti; lyhyemmän isoformin ilmentymiskuvio on päinvastainen. Eri isoformien differentiaalisella ilmentymisellä vaikuttaisi olevan rooli simpukan tonotopian eli paikasta riippuvan taajuusherkkyyden muodostumisessa \cite{hollo2023}.

Tässä työssä tutkittiin, mitä tietoa biotietokannat UniProtKB ja GOA (Gene Ontology Annotation) tarjoavat \textit{MYO7A}:sta sekä selvitettiin geenin evolutiivista konservaatiota 11 eläinlajin välillä.

\section{Menetelmät}
Tiedonhaun ja analyysin kaikki askeleet tehtiin Python-ohjelmointikielellä Jupyter-muistioon, joka on saatavilla osoitteessa \url{https://github.com/ogkristi/bioinfo1}.

Ihmisen MYO7A:sta etsittiin tietoa UniProtKB:sta käyttäen UniProtin verkkosivun REST API:a (https://rest.uniprot.org/uniprotkb/). Geenin Gene Ontology (GO) annotaatioita haettiin UniProt ID:n perusteella QuickGO:n REST API:n (https://www.ebi.ac.uk/QuickGO/services/) kautta. Löytyneistä GO-luokista valittiin satunnaisesti viisi ja niiden esivanhemmista muodostettu graafi noudettiin saman APIn kautta.

Ihmisen \textit{MYO7A} mRNA-sekvenssejä etsittiin Biopythonin v.1.83 \cite{biopython} Entrez-moduulilla NCBI:n RefSeq-tietokannasta. Haku rajattiin tuloksiin, joiden otsikossa ei esiinny "PREDICTED". Tuloksista valittiin pisin (NM\_000260), kanonisen MYO7A:n transkripti analyysiin. Sekvenssille etsittiin mRNA-homologeja kym\-menestä muusta lajista: kapusiiniapina, aasiannorsu, isoviiksisiippa, kaljurotta, kotihiiri, japaninviiriäinen, vuokkokala, sokkeloviuhkapyrstö, kirjokilpikonna ja vihreä merisiili (tieteelliset nimet taulukossa \ref{blast}). Haku tehtiin Biopythonin Blast-moduulilla käyttäen blastn-ohjelmaa ja rajaten haku refseq\_rna-tieto\-kantaan. Tuloksista valittiin jokaiselle lajille paras osuma ja niitä vastaavat sekvenssit noudettiin Biopythonin Entrez-moduulilla. Lajivalinnassa suosittiin vanhemisbiologiassa käytettyjä malliorganismeja. Valintaprosessi oli iteratiivinen, sillä esimerkiksi grönlanninhaille ei löytynyt RefSeq:stä mRNA-sekvenssejä ja savanninorsulle paras osuma edusti \textit{MYO7B}:tä. 

Sekvenssit rinnastettiin MAFFT v.7.520 \cite{mafftv7} ohjelmiston iteratiivisella L-INS-i algoritmilla käyttäen oletusasetuksia. Rinnastuksen visualisointiin käytettiin Python-pakettia pyMSAviz v.0.4.2. Sitten usean sekvenssin rinnastuksesta muodostettiin fylogeneettinen puu suurimman uskottavuuden menetelmään perustuvalla IQ-TREE v.2.2.6 \cite{iqtree} ohjelmistolla käyttäen 1000 UFBoot-toistoa ja määräten merisiili ulkoryhmäksi. Puu visualisoitiin Biopythonin Phylo-moduulilla \cite{phylo}.


\section{Tulokset}
\textbf{UniProtKB.} MYO7A kuuluu TRAFAC luokan myosiini-kinesiini ATPaasi superperheeseen ja edelleen myosiinien perheeseen. Kanoninen MYO7A on 2215 aminohappoa pitkä ja koostuu domeeneista myosiinimoottori, IQ 1-5, MyTH4 1, FERM 1, SH3, MyTH4 2 ja FERM 2. UniProtKB tunnustaa proteiinille 8 isoformia, mutta kommentoi, että niitä lienee todellisuudessa enemmän. Proteiinissa on fosforyloitavia seriini- ja treoniinitähteitä.

MYO7A osallistuu solunsisäiseen liikenteeseen sitoutumalla häntäosallaan kalvomaisiin osastoihin ja liikuttamalla niitä aktiinifilamentteja pitkin. Sillä on tärkeitä tehtäviä verkkokalvossa: valoreseptorissa se osallistuu kalvolevyjen uudistamiseen ja opsiinin kuljetuksen sääntelyyn. Pigmenttiepiteelissä se osallistuu melanosomien ja fagosomien sijoitteluun. Sisäkorvassa sillä on rooleja karvasolukimppujen erilaistumisessa, morfogeneesissä ja järjestäy\-tymisessä. Se osallistuu karvasoluissa ototoksisten aminoglykosidien kalvorakkulavälitteiseen liikennöintiin. MYO7A, USH1C, USH1G ja CDH23 muodostavat yhdessä verkoston, joka välittää mekanotransduktiota karvasoluissa. MYO7A on välttämätön normaalille kuulolle.

\textbf{QuickGO.} \textit{MYO7A}:lle löytyi 59 ainutlaatuista GO-luokkaa. Molekylaariseen funktioon liittyvät luokat kertovat sen sitovan seuraavia molekyylejä: ATP, ADP, kalmoduliini, aktiini ja spektriini. Solun komponentteja, joihin se voi sijoittua, ovat sukakarva, mikrovillus, melanosomi, valoreseptorin ulompi ja sisempi jaoke sekä yhdysvärekarva, synapsi, solun korteksi ja lysosomin kalvo. Esimerkkejä biologisista prosesseista, joihin se osallistuu, ovat solunsisäinen proteiinien kuljetus, pigmenttijyväsen kuljetus, fagolysosomien kokoonpano, aktiinisäikeiden järjestäytyminen, näkeminen, kuuleminen ja tasapainoaisti. Lisäksi biologisissa prosesseissa tulee esiin korvan ja silmän aistinsolujen erilaistuminen ja morfogeneesi. Viiden GO-luokan esivanhemmista muodostettu graafi on liitekuvassa \ref{gograafi}.

\textbf{Fylogenia.} Taulukossa \ref{blast} on BLAST-osumista analyysiin valittujen sekvenssien tunnisteet ja pituudet. Kaikkien osumien E-arvo oli käytännössä 0. Sekvenssien rinnastuksesta on 800 emäksen pala aloituskodonin ympäristöstä visualisoituna liitekuvassa \ref{msa}. Usean sekvenssin rinnastuksesta muodostettu puu on kuvassa \ref{puu}.

\begin{table}[!h]
	\centering
	\caption{Analyysiin valitut sekvenssit.} \label{blast}
	\begin{tabular}{llrc}
	\toprule
	\textbf{Organismi} & \textbf{Accession} & \textbf{Pituus} \\
	\midrule
	\textit{Homo sapiens} & NM\_000260 & 7483 \\
	\textit{Cebus imitator} & XM\_017533058 & 7372 \\
	\textit{Heterocephalus glaber} & XM\_013067417 & 7318 \\
	\textit{Elephas maximus} & XM\_049889740 & 9948 \\
	\textit{Myotis brandtii} & XM\_014540388 & 9541 \\
	\textit{Mus musculus} & NM\_001256081 & 7481 \\
	\textit{Coturnix japonica} & XM\_015852500 & 7484 \\
	\textit{Chrysemys picta} & XM\_042849556 & 8349 \\
	\textit{Amphiprion ocellaris} & XM\_055012531 & 7688 \\
	\textit{Nothobranchius furzeri} & XM\_054734303 & 12445 \\
	\textit{Lytechinus variegatus} & XM\_041626427 & 7119 \\
	\bottomrule
	\end{tabular}
\end{table}

\begin{figure}[!h]
	\centering
	\includegraphics[width=\textwidth]{tree.png}
	\caption{IQ-TREE:n tuottama suurimman uskottavuuden puu. Haarojen bootstrap-tukiarvot ovat harmaalla. Oksan pituus merkitsee odotettua emässubstituutioiden lukumäärää per sijainti.} \label{puu}
\end{figure}

\section{Tulosten tarkastelu}
Valtaosa analyysiin hyväksytyistä mRNA-sekvensseistä on ennustuksia, koska harvalle lajille löytyi todellisten transkriptien sekvenssejä eikä analyysia haluttu rajoittaa tähän suppeaan lajivalikoimaan. Sekvenssien pituuksissa on jonkun verran hajontaa; ääripäässä \textit{N. furzeri}n sekvenssi on noin viisi tuhatta emästä pidempi kuin ihmisen. MAFFT L-INS-i käyttää sisäisesti paikallista rinnastusta ja dokumentaation mukaan se suoriutuu hyvin, vaikka sekvensseissä olisi huonosti rinnastuvia terminaalisia osuuksia. Tästä johtuen sekvenssejä ei suljettu ulos pituuserojen takia. Rinnastuksen visualisoinnista (kuva \ref{msa}) voi nähdä, että rinnastus on järjellinen aloituskodonista eteenpäin.

Fylogeneettisessä puussa kalat (\textit{A. ocellaris} ja \textit{N. furzeri}), nisäkkäät sekä lintu ja kilpikonna yhdessä (\textit{C. japonica} ja \textit{C. picta}) muodostavat korkealla luottamuksella kukin omat kladinsa. Luottamus on korkea myös kädellisten omaan kladiin (\textit{H. sapiens} ja \textit{C. imitator}). Sen sijaan nisäkkäiden alipuun sisäinen haarautumisjärjestys on epävarma, mihin voi vaikuttaa esimerkiksi se, että kaljurotan ja lepakoiden aistit ovat hyvin erikoistuneet. Puussa ihminen ja kapusiini ovat lähimmät sukulaiset. Lintu ja kilpikonna ovat melko kaukaisia sukulaisia samaan kladiin sijoittumisesta huolimatta. Merisiili (\textit{L. variegatus}) on kaukaisinta sukua kaikille muille.

Geeni on selvästi alkuperältään todella vanha, sillä sille löytyi ortologi merisiilestä, jolla ei ole varsinaisia silmiä tai korvia. Puun topologiassa saattaa näkyä eläimien aistielimien erilaiset adaptaatiot. 

\printbibliography[title={Kirjallisuusluettelo}]

\renewcommand{\thefigure}{A\arabic{figure}}
\setcounter{figure}{0}
\begin{appendices}
\renewcommand\thefigure{\thesection.\arabic{figure}} 
\section{Liitteet}

\begin{figure}[!h]
	\centering
	\includegraphics[width=\textwidth]{gochart.png}
	\caption{GO-graafi.} \label{gograafi}
\end{figure}

\begin{figure}[!h]
	\centering
	\includegraphics[width=\textwidth]{msa.png}
	\caption{Osa usean sekvenssin rinnastuksesta. Ihmisen aloitusmetioniini on merkattuna kohdassa 5316.} \label{msa}
\end{figure}

\end{appendices}

\end{document}